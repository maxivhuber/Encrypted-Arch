\documentclass[11pt,a4paper]{article}
\usepackage[utf8]{inputenc}
\usepackage[german]{babel}
\usepackage[T1]{fontenc}
\usepackage{amsmath}
\usepackage{amsfonts}
\usepackage{amssymb}
\usepackage{graphicx}
\usepackage{color}

\author{Maximilian Huber}
\title{Arch-Linux}
\date{\today}
\setlength{\parindent}{0pt}

\begin{document}

\begin{titlepage}
\maketitle
\thispagestyle{empty}
\end{titlepage}

\tableofcontents
\thispagestyle{empty}	
\cleardoublepage
\setcounter{page}{1}

\section{Partitionierung}
\begin{table}[h]
\begin{tabular}{|p{3cm}|p{3cm}|p{3cm}| p{3cm} |}
\hline
Mount point on the installed system & Partition & Partition type   &  Suggested size \\ \hline
/boot/efi & /dev/sda1  & EFI system partition (code:ef00)  & 512M \\ \hline
swap & /dev/sda2  & Linux swap(code:8200)  & RAM > 8GB? mind. 4GB : RAM*2 \\ \hline
subvolid=5 & /dev/sda3  & Luks Filesystem (code:8309)  & rest  \\ \hline
\end{tabular}
\caption{Empfohlenes Layout}
\end{table}

Das Filesystem und die swap Partition werden in einen verschlüsselten Container gepackt!\\

Unsere Verschlüsselte System Partition bekommt ein Flaches BTRFS-Layout (vgl. https://btrfs.wiki.kernel.org/index.php/SysadminGuide\#Layout)

Je nach Alter müssen eventuell kleinere Dinge abgeändert werden wie z.B. die Pakete.

Es sollten unbedingt aktuelle Verschlüsselungsalgorithmen benutzt werden!
\section{Formatierung}
\subsection{Vorweg}
Wenn ihr in die Live-Umgebung bootet, ist zu empfehlen sich mit ssh auf den Rechner aufzuschalten, um Befehle mit copy\&paste einzufügen.
Dafür müsst ihr in der Live-Umgebung ein Passwort für root setzen und den sshd Service starten.
\subsection{Formatierung}
\begin{itemize}
\item EFI-System Partition:

mkfs.vfat -F 32 -n EFI /dev/sdx

\item Linux Dateisystem:
\begin{enumerate}
\item cryptsetup luksFormat -{}-\textcolor{red}{\text{type luks1 -h sha512 -s 512 -{}-iter-time 5000}}  /dev/sdx
\item cryptsetup luksOpen /dev/sdx 	cryptroot
\item mkfs.btrfs -L arch-crypt /dev/mapper/cryptroot
\item mount /dev/mapper/cryptroot /mnt
\item btrfs subvolume create /mnt/@
\item btrfs subvolume create /mnt/@home
\item btrfs subvolume create /mnt/@cache
\item btrfs subvolume create /mnt/@log
\item btrfs subvolume create /mnt/@spool
\item btrfs subvolume create /mnt/@temp
\item btrfs subvolume create /mnt/@snapshots
\item btrfs subvolume create /mnt/@srv
\item ls /mnt (zeigt die erstellten Verzeichnisse an)
\item umount /mnt
\item mount -o compress=zstd,subvol=@ /dev/mapper/cryptroot /mnt
\item mkdir -p /mnt/\{home,.snapshots,var/cache,var/log,/var/spool,var/temp,boot/efi,btrfs,srv\}
\item mount -o compress=zstd,subvol=@home /dev/mapper/cryptroot /mnt/home/
\item mount -o compress=zstd,subvol=@snapshots /dev/mapper/cryptroot /mnt/.snapshots
\item mount -o compress=zstd,subvol=@cache /dev/mapper/cryptroot /mnt/var/cache
\item mount -o compress=zstd,subvol=@log /dev/mapper/cryptroot /mnt/var/log
\item mount -o compress=zstd,subvol=@spool /dev/mapper/cryptroot /mnt/var/spool
\item mount -o compress=zstd,subvol=@temp /dev/mapper/cryptroot /mnt/var/temp
\item mount -o compress=zstd,subvol=@srv /dev/mapper/cryptroot /mnt/srv
\item mount -o compress=zstd,subvolid=5 /dev/mapper/cryptroot /mnt/btrfs (optional! stellt nur das top-level btrfs im System da )
\item mount /dev/sda1 /mnt/boot/efi
\item df -Th (Überprüfung)
\end{enumerate}
\end{itemize}
\section{Basis Betriebssystem installieren}
Befehl:

pacstrap /mnt base base-devel linux linux-headers \textcolor{red}{\text{linux-hardened linux-hardened-headers}} linux-firmware  \textcolor{red}{\text{ intel-ucode}} bash-completion btrfs-progs dosfstools grub efibootmgr dhcpcd nano cryptsetup
\section{Einrichten des Betriebssystems}
\begin{itemize}
\item genfstab -U /mnt >{}> /mnt/etc/fstab
\item arch-chroot /mnt
\item fstab nachberarbeiten (https://wiki.archlinux.de/title/Arch\_auf\_BtrFS\#fstab\_bearbeiten)
\item echo hostname > /etc/hostname
\item echo LANG=de\_DE.UTF-8 > /etc/locale.conf
\item echo LANGUAGE=de\_DE >{}> /etc/locale.conf
\item echo KEYMAP=de-latin1 > /etc/vconsole.conf
\item ln -s /usr/share/zoneinfo/Europe/Berlin /etc/localtime
\item useradd -m -g users -G wheel,audio,video -s /bin/bash username
\item passwd \glqq{}name\grqq{}
\item EDITOR=nano visudo

Einkommentieren: \%wheel ALL=(ALL) ALL
\item nano /etc/locale.gen

Einkommentieren: de\_DE.UTF-8
\item locale-gen

\end{itemize}
\section{Installation und Einrichtung der GUI}
\begin{itemize}
\item pacman -S xorg-server xorg-xinit xorg-server-xephyr ttf-dejavu \textcolor{red}{\text{xfce4 xfce4-goodies}} lightdm \textcolor{red}{\text{ lightdm-webkit2-greeter nvidia}}  networkmanager network-manager-applet nm-connection-editor alsa alsa-tools alsa-utils pulseaudio-alsa pavucontrol cronie ntp avahi acpid
\item in \glqq{}/etc/lightdm/lightdm.conf\grqq{} \textbf{unter} [Seat:*] 

greeter-session=\textcolor{red}{\text{ lightdm-webkit2-greeter}} 

hinzufügen.
\item systemctl enable dhcpcd acpid avahi-daemon cronie ntpd NetworkManager lightdm
\item ntpd -gq
\item localectl set-x11-keymap de (Wenn nicht möglich kann dieser Schritt ohne Probleme verschoben werden)
\end{itemize}
\subsection{Gnome Pakete}
Möchten wir eine Gnome Installation, können wir anstatt der xfce4 Pakete, die Paketgruppe \textcolor{red}{\text{gnome }} installieren.

Wichtige Pakete die man auswählen sollte sind:\\

eog evince file-roller gnome-control-center gnome-disk-utility gnome-font-viewer gnome-keyring gnome-menus gnome-screenshot gnome-settings-daemon gnome-shell gnome-system-monitor gnome-terminal  gnome-themes-extra gvfs mousetweaks mutter nautilus rygel tracker xdg-user-dirs-gtk gnome-software 
\section{Systemkonfiguration }
\begin{itemize}
\item Wir müssen die UUID unserer verschlüsselten Partition herausfinden
\item blkid
\item blkid /dev/sdx >{}> /etc/default/grub
\item Fortfahren mit Video 1 (15:58 - 16:49)
\item grub-install -{}-target=x86\_64-efi -{}-efi-directory=/boot/efi -{}-bootloader-id=arch-crypt 
\item grub-mkconfig -o /boot/grub/grub.cfg
\item dd bs=512 count=4 if=/dev/random of=/crypto\_keyfile.bin iflag=fullblock
\item chmod 600 /crypto\_keyfile.bin
\item chmod 600 /boot/initramfs-linux*
\item cryptsetup luksAddKey /dev/sdx /crypto\_keyfile.bin
\item nano /etc/mkinitcpio.conf

Fortfahren mit Video 1 (21:11 - 22:37)
\item FSCK HOOK kann entfernt werden da btrfs eigene Tools besitzt z.B. btrfs.scrub
\item mkinitcpio -p linux
\item exit
\item umount -R /mnt
\item cryptsetup luksClose cryptroot
\item reboot
\end{itemize}
Folgende Befehle können benutzt werden um den Auto-Login in lightdm zu aktivieren(nur wenn eine Person den PC benutzt):
\begin{itemize}
\item nano /etc/lightdm/lightdm.conf

Unter [Seat:*] kann autologin-user=username einkommentiert werden.
\item groupadd -r autologin
\item gpasswd -a username autologin 
\end{itemize}
\section{Verschlüsselter Swap}
https://wiki.archlinux.org/index.php/Dm-crypt/Swap\_encryption\#With\_suspend-to-disk\_support
\begin{itemize}
\item cryptsetup luksFormat -{}-\textcolor{red}{\text{type luks1 -h sha512 -s 512 -{}-iter-time 5000}}  /dev/sdx

\item cryptsetup open /dev/sdx swapDevice
\item mkswap -L SWAP /dev/mapper/swapDevice
\item Dateien in /etc/initcpio/*/openswap anlegen.

nach: https://wiki.archlinux.org/index.php/Dm-crypt/Swap\_encryption\#mkinitcpio\_hook
\item dd bs=512 count=4 if=/dev/urandom of=/etc/keyfile-cryptswap.bin
\item chmod 600 /etc/keyfile-cryptswap.bin
\item cryptsetup luksAddKey /dev/sdx /etc/keyfile-cryptswap.bin
\item blkid /dev/mapper/swapDevice >{}> /etc/fstab
\item Fortfahren mit Video 2 (5:33 - 6:05)
\item swapon -a
\item nano /etc/mkinitcpio.conf 
\item Fortfahren mit Video 2 (6:17 - 6:45)
\item blkid /dev/mapper/swapDevice >{}> /etc/default/grub
\item nano /etc/default/grub
\item Fortfahren mit Video 2 (7:08 - 7:20)
\item Dateien in /etc/initcpio/*/openswap anpassen
\item grub-mkconfig -o /boot/grub/grub.cfg
\item mkinitcpio -p linux 
\end{itemize}
Wenn wir in GNOME den Computer Hibernaten lassen wollen, müssen wir im Menü zum ausschalten den ALT Key drücken.
\section{Snapshots}
\subsection{Snapshot manuell}
\begin{itemize}
\item mount -o compress=zstd,subvolid=5 /dev/mapper/cryptroot /btrfs
\item ls /btrfs (Es werden alle subol. angezeit weil wir das top-level-btrfs gemountet haben)
\item btrfs subvolume snapshot -r /btrfs/@ /btrfs/@snapshots/@\$(date +\%Y-\%m-\%d)

Ein Snapshot von @ (also /) wird nach @snapshots/date gebracht (also /.snapshots/date)
\end{itemize}
\subsection{Snapshot löschen}
Jeder Snapshot ist gleichzeitig ein Subvolume!

btrfs subvolume delete /btrfs/@snapshots/@2019-12-30

\subsection{Snapper}
\subsubsection{Einrichten}
\begin{itemize}
\item umount /.snapshots
\item rmdir /.snapshots/
\item snapper -c root create-config /
\item btrfs subvolume delete /.snapshots/
\item mkdir /.snapshots
\item chmod 750 /.snapshots
\item mount -a
\item nano /etc/snapper/configs/root

Konfiguration von Snapper anpassen
\subsubsection{Anders}
Snapper wird in .snapshots automatisch snapshots anlegen die man mit den obigen Kommandos bearbeiten kann.
Es können auch weitere tools wie z.B. snap-pac, snap-sync genutzt werden.
\end{itemize}

\subsection{Snapshot wiederherstellen}
\begin{itemize}
\item https://wiki.archlinux.org/index.php/Snapper\#Restoring\_/\_to

\_a\_previous\_snapshot\_of\_@
\item top level btrfs mounten!
\item Wenn @broken schon existiert können wir das Verzeichnis löschen

rm -R /btrfs/@broken

Das subvolume wird automatisch entfernt
\item mv /btrfs/@ /btrfs/@broken
\item btrfs subvol snapshot /btrfs/@snapshots/\#/snapshot /btrfs/@
\item reboot
\end{itemize}
Möchten wird das jetzt nicht mehr aktive / Verzeichnis wieder nutzen, können wir das folgendermaßen tun:
\begin{itemize}
\item mv /btrfs/@ /btrfs/@broken
\item /btrfs/@broken /btrfs/@
\item reboot
\end{itemize}

\section{System Emergency Resuce}
Ist das System nicht mehr Bootbar müssen wir von einer Live-CD(https://www.archlinux.de/download) booten und das @ subvolume austauschen.
Dafür müssen wir folgende Schritte durchführen.
\begin{itemize}
\item loadkeys de
\item cryptsetup luksOpen /dev/sda3 	cryptroot
\item mount /dev/mapper/cryptroot /mnt
\item nano /mnt/@snapshots/*/info.xml
\item mv /mnt/@ /mnt/@broken
\item btrfs subvol snapshot /mnt/@snapshots/\#/snapshot /mnt/@
\item umount /mnt
\item reboot
\end{itemize}
Nach einem erfolgreichem Reboot kann das neue subvolume @broken entfernen:
\begin{itemize}
\item Top Level btrfs mounten
\item rm -R /btrfs/@broken
\end{itemize}



\section{Wichige Programme}
\begin{itemize}
\item Repos:

thunderbird-i18n-de firefox-i18n-de chromium intellij-idea-community-edition texmaker texlive-most hunspell-de veracrypt discord jdk-openjdk vlc rhythmbox virtualbox virtualbox-guest-iso code transmission-gtk flatpak gradle seahorse filezilla reflector gnome-chrome-shell plank gnome-tweaks neofetch
\item Flatpak:

Dropbox, Android Studio, Sublime, Steam, Video Downloader
\item AUR:

Github-Desktop, Cryptomator, Moodle-Desktop, Joplin(braucht nodejs und npm), Lastpass

Tipp: makepkg -sic

\item Empfohlene Gnome Extensions

Impatience, NetSpeed, Arch Linux Updates Indicator, Show Applications, User Themes, Hibernate Status Button, Alternate Tab

\item List of applications:

https://wiki.archlinux.org/index.php/list\_of\_applications

\end{itemize}

\section{Video Referenzen}
\subsection{Video 1}
https://www.youtube.com/watch?v=OTrZcIG4gDE
\subsection{Video 2}
https://www.youtube.com/watch?v=yMqWrt17Z18

\end{document}